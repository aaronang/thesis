\documentclass[twoside,a4paper,11pt]{memoir}
\usepackage{times}
\usepackage{pslatex}
\usepackage{url}
\usepackage{mscthesis}
\let\newfloat\undefined \usepackage{algorithmic}
\usepackage{algorithm}
\usepackage{graphicx}
\usepackage{color}
\usepackage{adjustbox}
\usepackage{xspace}
\usepackage{enumitem}
\usepackage{multirow}
\usepackage{subcaption}
\usepackage[hidelinks]{hyperref}
\usepackage{cleveref}
\usepackage[justification=centering]{caption}
\usepackage[table,xcdraw]{xcolor}
\definecolor{Gray}{gray}{0.85}

\captionsetup[subfigure]{subrefformat=simple,labelformat=simple}
\renewcommand\thesubfigure{(\alph{subfigure})}

\usepackage{subcaption}
\usepackage{mwe}

\newcommand{\etal}{\emph{et al.}\xspace}
\newcommand\todo[1]{\noindent\textcolor{red}{TODO: #1}}
\newcommand\note[1]{\noindent\textcolor{red}{NOTE: #1}}

\def\chapterautorefname{Chapter}
\def\sectionautorefname{Section}
\def\subsectionautorefname{Section}


\title{DDU}
\subtitle{Master's Thesis} 

\author{Aaron Ang}
\authoremail{\url{a.w.z.ang@student.tudelft.nl}}
\birthplace{Amstelveen, The Netherlands}
\studentid{4139194}

\company{
\includegraphics[height=1.3cm]{figures/parc_logo.png}\\
PARC, a Xerox company\\
3333 Coyote Hill Road\\
Palo Alto, CA 94304\\
\url{www.parc.com}
}

\chair{Prof. Dr. A. van Deursen, Faculty EEMCS, TU Delft}
\supervisor{Prof. Dr. A. van Deursen, Faculty EEMCS, TU Delft}
\externalsupervisor{Prof. Dr. R. Maranhao, University of Lisbon}

\setcounter{tocdepth}{1}
\setsecnumdepth{subsection}
\maxsecnumdepth{subsection}

\begin{document}

\frontmatter
\thispagestyle{empty}
\maketitle
\makeformaltitlepages{hello}

\chapter{Preface}
\label{ch:preface}

This is where you thank people for helping you etc.


\vskip1cm
\begin{flushright}
\theauthor\\
Delft, The Netherlands \\
\today\\
\end{flushright}



\cleardoublepage\tableofcontents
\cleardoublepage\listoffigures
\cleardoublepage\mainmatter

\chapter{Introduction}
\label{ch:introduction}
Software systems are complex and error-prone, likely to expose failures to the end user.
When a failure occurs, the developer has to debug the system to eliminate the failure.
This debugging process can be described in three phases \cite{parnin2011automated}.
In the first phase, the developer has to pinpoint the fault, also known as the root cause, in code that causes the failure.
In the second phase, the developer has to develop an understanding of the root cause and its context.
Finally, in the third phase, the developer has to implement a patch that corrects the behavior of the system.
This process is time-consuming and can account for 30\% to 90\% of the software development cycle \cite{robbins2003debugging, beizer2003software, britton2013reversible}. 

Traditionally, developers use four different approaches to debug a software system, namely program logging, assertions, breakpoints and profiling \cite{wong2016survey}.
Program logging is the act of inserting \emph{print} statements in the code to observe program state information during execution.
Assertions are constraints that can be added to a program that have to evaluate to true during execution time. 
Breakpoints allow the developer to pause the software system during execution, and observe and modify variable values.
Profiling is used to perform runtime analysis and collect metrics on, for example, execution speed and memory usage.
These techniques provide an intuitive approach to localize the root cause of a failure, but, as one might expect, are less effective in the massive size and scale of software systems today.

Therefore, in the last decades a lot of research has been performed on improving and developing \emph{advanced} fault localization techniques \cite{wong2016survey} such that they are applicable to the software systems of today.
Specifically, a prominent fault localization technique is spectrum-based fault localization (SBFL).
SBFL techniques pinpoint faults in code based on execution information of a program, also known as a program spectrum \cite{reps1997use}.
It does this by outputting a list of suspicious components, for example statements or methods, ranked by their suspiciousness.
Intuitively, if a statement is executed primarily during failed executions, then this statement might be assigned a higher suspiciousness score. 
Similarly, if a statement is executed primarily during successful executions, then this statement might be assigned a lower suspiciousness score.

While SBFL techniques are promising for debugging purposes, these techniques are dependent on the quality of a test suite.
Currently, test suites are optimized with respect to adequacy measurements that focus on error detection, e.g. branch coverage, line coverage.
However, Perez \etal~\cite{DBLP:conf/icse/PerezAD17} show evidence that optimizing a test suite with respect to DDU --- a metric to quantify the test suite's diagnosability --- improves the diagnostic performance of SBFL by 34\% compared to a test suite optimized with respect to branch coverage.
The goal of DDU is to capture diagnosability and to serve as a complementary metric to code coverage for developers to use to improve the test suite's diagnosability.

\section{Problem Definition}
\label{sec:problem_definition}
Currently, when the DDU is computed for a given test suite, its value is in the domain $[ 0, 1 ]$, where $0$ suggests that the test suite's diagnosability is low, and $1$ suggests that the test suite's diagnosability is high.
The problem with this value is that the developer does not know how to extend or update the test suite given a DDU value.
For example, when the test suite's DDU is equal to $0.1$, the developer does not know how to write tests that improve the DDU.
In other words, time spent on software debugging cannot be reduced using DDU because its practical implications are unclear to the developer.

\section{Goal}
Although DDU is currently not usable in practice, Perez \etal \cite{DBLP:conf/icse/PerezAD17} have shown that optimizing a test suite with respect to DDU can yield a 34\% gain in diagnostic performance using SBFL.
In addition, having a test suite with a high diagnosability could possibly reduce the time spent debugging because the fault is easier to find manually.
Therefore, the goal of this thesis is to find ways to make DDU usable in practice.
In other words, we explore possibilities to convey DDU to the developer such that the developer knows what kind of tests to write to improve the system's diagnosability.

\section{Structure of Report}
The structure of this report is as follows. 
\todo{Update once all chapters are done.}

\chapter{Background}
\label{ch:background}

In this chapter, we discuss topics that are relevant to understanding the following chapters.
First, we discuss spectrum-based reasoning, which is used in the experiments to compute the diagnostic performance.
Second, we discuss the metric used to evaluate the diagnostic performance of spectrum-based fault localization techniques.
Then, we explain the definition of diagnosability and diagnosability metrics.
Finally, we briefly discuss error detection.

\section{Spectrum-Based Reasoning (SBR)}
\todo{Describe how SR works and Barinel.}

Spectrum-based reasoning is a spectrum-based fault localization technique that leverages a Bayesian reasoning framework to diagnose fault candidates that could potentially be the root cause for a given software failure \cite{abreu2009spectrum}.
In SBR, we define a finite set $\mathcal{C} = \langle c_1, c_2, \ldots, c_M \rangle$ of $M$ system components, and the finite set $\mathcal{T} = \langle t_1, t_2, \ldots, t_N \rangle$ of $N$ system transactions, such as test executions.
The outcomes of all system transactions are defined as an error vector $e = \langle e_1, e_2, \ldots, e_N \rangle$, where $e_i = 1$ indicates that transaction $t_i$ has failed and $e_i = 0$ otherwise.
To keep track of which system components were executed during which system transactions, we construct a $N \times M$ activity matrix $\mathcal{A}$, where $\mathcal{A}_{ij} = 1$ indicates that component $c_j$ was hit during transaction $t_i$.
The pair $(\mathcal{A}, e)$ is also known as a program spectrum, which was first coined by Reps \etal \cite{reps1997use}.

SBR distinguishes itself from SBFL techniques by leveraging a reasoning framework.
More specifically, the diagnostic report is generated by reasoning about the program spectrum instead of using a so-called similarity coefficient.

\section{Evaluation of Diagnosis} 
\todo{Describe wasted effort}

\section{Diagnosability}
\todo{Explain concept and definition of diagnosability: the property of faults to be easily and precisely located}

\subsection{Diagnosability Metric: Entropy}
\subsection{Diagnosability Metric: DDU}

\subsubsection{Density}

\subsubsection{Diversity}

\subsubsection{Uniqueness}

\section{Error detection}

\chapter{Research Questions}
\label{ch:research_questions}
To explore possibilities to convey DDU to the developer with usability in mind, we define four research questions that are relevant to investigate.
Note that the nature of this study will be exploratory because little research has been performed on software diagnosability.
Moreover, DDU is a metric that has been proposed recently and, therefore, no research has investigated this metric.

To make recommendations based on DDU, it is necessary to obtain a better understanding of DDU and its individual components: density, diversity, and uniqueness.
Specifically, we are interested in what common values are for DDU and its individual components in open source software.
Hence, the first research question is defined as follows.

\begin{framed}
\noindent
\textbf{RQ1:} What kind of values do density, diversity, uniqueness, and DDU take on in open source software?
\end{framed}

Next, we would like to validate that DDU and diagnosability are positively correlated, i.e. the higher DDU, the better the diagnosability, and vice versa.

\begin{framed}
\noindent
\textbf{RQ2:} What is the relation between density, diversity, uniqueness, and DDU and diagnosability?
\end{framed}

\begin{framed}
\noindent
\textbf{RQ3:} What is the relation between density, diversity, uniqueness, and DDU and test coverage?
\end{framed}

\begin{framed}
\noindent
\textbf{RQ4:} What kinds of tests have positive or negative effects on density, diversity, uniqueness, and DDU?
\end{framed}

\chapter{RQ1}
\label{ch:rq1}
In this chapter, we discuss

\chapter{Conclusion}
\label{ch:conclusion}

\bibliographystyle{plain}
\bibliography{bibliography}

\end{document}
